\documentclass[]{elsarticle} %review=doublespace preprint=single 5p=2 column
%%% Begin My package additions %%%%%%%%%%%%%%%%%%%
\usepackage[hyphens]{url}
\usepackage{lineno} % add
\providecommand{\tightlist}{%
  \setlength{\itemsep}{0pt}\setlength{\parskip}{0pt}}

\bibliographystyle{elsarticle-harv}
\biboptions{sort&compress} % For natbib
\usepackage{graphicx}
\usepackage{booktabs} % book-quality tables
%% Redefines the elsarticle footer
%\makeatletter
%\def\ps@pprintTitle{%
% \let\@oddhead\@empty
% \let\@evenhead\@empty
% \def\@oddfoot{\it \hfill\today}%
% \let\@evenfoot\@oddfoot}
%\makeatother

% A modified page layout
\textwidth 6.75in
\oddsidemargin -0.15in
\evensidemargin -0.15in
\textheight 9in
\topmargin -0.5in
%%%%%%%%%%%%%%%% end my additions to header

\usepackage[T1]{fontenc}
\usepackage{lmodern}
\usepackage{amssymb,amsmath}
\usepackage{ifxetex,ifluatex}
\usepackage{fixltx2e} % provides \textsubscript
% use upquote if available, for straight quotes in verbatim environments
\IfFileExists{upquote.sty}{\usepackage{upquote}}{}
\ifnum 0\ifxetex 1\fi\ifluatex 1\fi=0 % if pdftex
  \usepackage[utf8]{inputenc}
\else % if luatex or xelatex
  \usepackage{fontspec}
  \ifxetex
    \usepackage{xltxtra,xunicode}
  \fi
  \defaultfontfeatures{Mapping=tex-text,Scale=MatchLowercase}
  \newcommand{\euro}{€}
\fi
% use microtype if available
\IfFileExists{microtype.sty}{\usepackage{microtype}}{}
\ifxetex
  \usepackage[setpagesize=false, % page size defined by xetex
              unicode=false, % unicode breaks when used with xetex
              xetex]{hyperref}
\else
  \usepackage[unicode=true]{hyperref}
\fi
\hypersetup{breaklinks=true,
            bookmarks=true,
            pdfauthor={},
            pdftitle={Social contexts shape information gathering, attention, and memory during familiar language comprehension and novel word learning},
            colorlinks=true,
            urlcolor=blue,
            linkcolor=magenta,
            pdfborder={0 0 0}}
\urlstyle{same}  % don't use monospace font for urls
\setlength{\parindent}{0pt}
\setlength{\parskip}{6pt plus 2pt minus 1pt}
\setlength{\emergencystretch}{3em}  % prevent overfull lines
\setcounter{secnumdepth}{0}
% Pandoc toggle for numbering sections (defaults to be off)
\setcounter{secnumdepth}{0}
% Pandoc header


\usepackage[nomarkers]{endfloat}

\begin{document}
\begin{frontmatter}

  \title{Social contexts shape information gathering, attention, and memory
during familiar language comprehension and novel word learning}
    
  \begin{abstract}
  
  \end{abstract}
  
 \end{frontmatter}

\section{Dissertation overview}\label{background}

Learning a first language should be hard. Consider that even concrete
nouns are often used in complex contexts with multiple possible
referents, which in turn have many conceptually natural properties that
a speaker could talk about. This ambiguity creates the potential for an
(in principle) unlimited amount of referential uncertainty in the
learning task.\footnote{This problem is a simplified version of Quine's
  \textit{indeterminacy of reference} (Quine 1960): That there are many
  possible meanings for a word (``Gavigai'') that include the referent
  (``Rabbit'') in their extension, e.g., ``white,'' ``rabbit,''
  ``dinner.'' Quine's broader philosophical point was that different
  meanings (``rabbit'' and ``undetached rabbit parts'') could actually
  be extensionally identical and thus impossible to tease apart.}.
Moreover, finding meaning in language requires rapdily establishing
reference during real-time interaction where the incoming information is
dynamic, multimodal, and transient. Remarkably, word learning proceeds
despite this uncertainty, with estimates of adult vocabularies ranging
between 50,000 to 100,000 distinct lexical concepts (P. Bloom 2002). How
do learners infer and retain such a large variety of word meanings from
data with this kind of ambiguity?

Social-pragmatic theories of language acquisition emphasize the
importance of the social context for word learning (P. Bloom 2002; Clark
2009; Hollich et al. 2000). Experimental work has shown that even
children as young as 16 months prefer to map novel words to objects that
are the target of a speaker's gaze and not their own (Baldwin 1993). In
an analysis of naturalistic parent-child labeling events, Yu and Smith
(2012) found that young learners tended to retain labels that were
accompanied by clear referential cues, which served to make a single
object dominant in the visual field. And correlational studies have
demonstrated strong links between early intention-reading skills (e.g.,
gaze following) and later vocabulary growth (Brooks and Meltzoff 2005;
Brooks and Meltzoff 2008; Carpenter et al. 1998). Moreover, studies
outside the domain of language acquisition have shown that the presence
of social cues: (a) produce better spatial learning of audiovisual
events (Wu et al. 2011), (b) boost recognition of a cued object
(Cleveland, Schug, and Striano 2007), and (c) lead to preferential
encoding of an object's featural information (Yoon, Johnson, and Csibra
2008).

Together, the evidence suggests that social information shapes the basic
information gathering processes that support word learning. My
dissertation work directly tests this information-seeking proposal. To
do this, I decompose the word learning problem into two components
(Frank, Lewis, and MacDonald 2016):

\begin{enumerate}
\def\labelenumi{\arabic{enumi}.}
\tightlist
\item
  language comprehension
\item
  disambiguation of reference
\end{enumerate}

In our previous work, we explored how children and adults gather visual
information from a social patner to faciliate rapid language
comprehension in American Sign Language (\protect\hyperlink{ch1}{Chapter
1}). and spoken language (\href{ch2}{Chapter 2}). Then I provide
evidence that social information modulates the representations
underlying cross-situational word learning (Chapter 3). Here, I propose
a study that will connect these two lines of research by testing
predictions of our information seeking account in novel word learning
contexts. The goal is to connect these two lines of research and
increase our understanding of how micro-processes that occur
in-the-moment of learning adapt to the value of seeking different kinds
of information in the social leanring context.

\section{Background literature}\label{background-literature}

The study of eye movements during language comprehension has provided
fundamental insights into the interaction between conceptual
representations of the world and the incoming linguistic signal. For
example, research shows that adults and children will rapidly shift
visual attention upon hearing the name of an object in the visual scene,
with a high proportion of shifts occurring prior to the offset of the
word (Allopenna, Magnuson, and Tanenhaus 1998; Tanenhaus et al. 1995).
Moreover, researchers have found that conceptual representations
activated by fixations to the visual world can modulate subsequent eye
movements during language processing (Altmann and Kamide 2007).

The majority of this work has used eye movements as a measure of the
output of the underlying language comprehension process, often using
linguistic stimuli that come from a disembodied voice. But in real world
contexts, people also gather information about the linguistic signal by
fixating on the language source. Consider a speaker who asks you to
``Pass the salt'' but you are in a noisy room, making it difficult to
understand the request. Here, comprehension can be facilitated by
gathering information via (a) fixations to the nonlinguistic visual
world (i.e., encoding the objects that are present in the scene) or (b)
fixations to the speaker (i.e., reading lips or perhaps the direction of
gaze).

But, this situation creates a tradeoff where the listener must decide
what kind of information to gather and at what time. How do we decide
where to look? We propose that people modulate their eye movements
during language comprehension in response to tradeoffs in the value of
gathering different kinds of information.

We test this adaptive tradeoff account using two case studies that
manipulate the value of different fixation locations for language
understanding: a) a comparison of processing sign vs.~spoken language in
children (E1), and b) a comparison of processing printed text vs.~spoken
language in adults (E2). Our key prediction is that competition for
visual attention will make gaze shifts away from the language source
less valuable than fixating the source of the linguistic signal, leading
people to generate fewer exploratory, nonlanguage-driven eye movements.

\section{Completed work}\label{completed-work}

\hypertarget{ch1}{\subsection{Chapter 1: Using eye movements to gather
information during real-time procesing of American Sign
Language}\label{ch1}}

\begin{center}\includegraphics[width=0.7\linewidth]{figs/unnamed-chunk-1-1} \end{center}

\begin{center}\includegraphics[width=0.7\linewidth]{figs/unnamed-chunk-2-1} \end{center}

\begin{center}\includegraphics[width=0.7\linewidth]{figs/unnamed-chunk-3-1} \end{center}

\begin{center}\includegraphics[width=0.7\linewidth]{figs/unnamed-chunk-4-1} \end{center}

\subsection{Chapter 2: An information seeking account of eye movements
in spoken and signed language comprehension}\label{ch2}

\begin{center}\includegraphics[width=0.7\linewidth]{figs/unnamed-chunk-5-1} \end{center}

\begin{center}\includegraphics[width=0.7\linewidth]{figs/unnamed-chunk-6-1} \end{center}

\subsection{Chapter 3: Social information modulates attention and memory
during word learning}\label{ch3}

\begin{center}\includegraphics[width=0.7\linewidth]{figs/unnamed-chunk-7-1} \end{center}

\begin{center}\includegraphics[width=0.7\linewidth]{figs/unnamed-chunk-8-1} \end{center}

\begin{center}\includegraphics[width=0.7\linewidth]{figs/unnamed-chunk-9-1} \end{center}

\begin{center}\includegraphics[width=0.7\linewidth]{figs/unnamed-chunk-10-1} \end{center}

\begin{center}\includegraphics[width=0.7\linewidth]{figs/unnamed-chunk-11-1} \end{center}

\section{Proposed work}\label{proposal}

\subsection{Eye movements for information seeking during novel word
learning}\label{eye-movements-for-information-seeking-during-novel-word-learning}

\section{References}\label{references}

\setlength{\parindent}{-0.3in} \setlength{\leftskip}{0.3in} \noindent

\hypertarget{refs}{}
\hypertarget{ref-allopenna1998tracking}{}
Allopenna, Paul D, James S Magnuson, and Michael K Tanenhaus. 1998.
``Tracking the Time Course of Spoken Word Recognition Using Eye
Movements: Evidence for Continuous Mapping Models.'' \emph{Journal of
Memory and Language} 38 (4). Elsevier: 419--39.

\hypertarget{ref-altmann2007real}{}
Altmann, Gerry, and Yuki Kamide. 2007. ``The Real-Time Mediation of
Visual Attention by Language and World Knowledge: Linking Anticipatory
(and Other) Eye Movements to Linguistic Processing.'' \emph{Journal of
Memory and Language} 57 (4). Elsevier: 502--18.

\hypertarget{ref-baldwin1993infants}{}
Baldwin, Dare A. 1993. ``Infants' Ability to Consult the Speaker for
Clues to Word Reference.'' \emph{Journal of Child Language} 20 (02).
Cambridge Univ Press: 395--418.

\hypertarget{ref-bloom2002children}{}
Bloom, Paul. 2002. \emph{How Children Learn the Meaning of Words}. The
MIT Press.

\hypertarget{ref-brooks2005development}{}
Brooks, Rechele, and Andrew N Meltzoff. 2005. ``The Development of Gaze
Following and Its Relation to Language.'' \emph{Developmental Science} 8
(6). Wiley Online Library: 535--43.

\hypertarget{ref-brooks2008infant}{}
---------. 2008. ``Infant Gaze Following and Pointing Predict
Accelerated Vocabulary Growth Through Two Years of Age: A Longitudinal,
Growth Curve Modeling Study.'' \emph{Journal of Child Language} 35 (01).
Cambridge Univ Press: 207--20.

\hypertarget{ref-carpenter1998social}{}
Carpenter, Malinda, Katherine Nagell, Michael Tomasello, George
Butterworth, and Chris Moore. 1998. ``Social Cognition, Joint Attention,
and Communicative Competence from 9 to 15 Months of Age.''
\emph{Monographs of the Society for Research in Child Development}.
JSTOR, i--174.

\hypertarget{ref-clark2009first}{}
Clark, Eve V. 2009. \emph{First Language Acquisition}. Cambridge
University Press.

\hypertarget{ref-cleveland2007joint}{}
Cleveland, Allison, Mariah Schug, and Tricia Striano. 2007. ``Joint
Attention and Object Learning in 5-and 7-Month-Old Infants.''
\emph{Infant and Child Development} 16 (3). Wiley Online Library:
295--306.

\hypertarget{ref-frank2016performance}{}
Frank, M, M Lewis, and Kyle MacDonald. 2016. ``A Performance Model for
Early Word Learning.'' In \emph{Proceedings of the 38th Annual
Conference of the Cognitive Science Society}.

\hypertarget{ref-hollich2000breaking}{}
Hollich, George J, Kathy Hirsh-Pasek, Roberta Michnick Golinkoff,
Rebecca J Brand, Ellie Brown, He Len Chung, Elizabeth Hennon, Camille
Rocroi, and Lois Bloom. 2000. ``Breaking the Language Barrier: An
Emergentist Coalition Model for the Origins of Word Learning.''
\emph{Monographs of the Society for Research in Child Development}.
JSTOR, i--135.

\hypertarget{ref-quine19600}{}
Quine, Willard V. 1960. ``0. Word and Object.'' \emph{111e MIT Press}.

\hypertarget{ref-tanenhaus1995integration}{}
Tanenhaus, Michael K, Michael J Spivey-Knowlton, Kathleen M Eberhard,
and Julie C Sedivy. 1995. ``Integration of Visual and Linguistic
Information in Spoken Language Comprehension.'' \emph{Science} 268
(5217). The American Association for the Advancement of Science: 1632.

\hypertarget{ref-wu2011infants}{}
Wu, Rachel, Alison Gopnik, Daniel C Richardson, and Natasha Z Kirkham.
2011. ``Infants Learn About Objects from Statistics and People.''
\emph{Developmental Psychology} 47 (5). American Psychological
Association: 1220.

\hypertarget{ref-yoon2008communication}{}
Yoon, Jennifer MD, Mark H Johnson, and Gergely Csibra. 2008.
``Communication-Induced Memory Biases in Preverbal Infants.''
\emph{Proceedings of the National Academy of Sciences} 105 (36).
National Acad Sciences: 13690--5.

\hypertarget{ref-yu2012embodied}{}
Yu, Chen, and Linda B Smith. 2012. ``Embodied Attention and Word
Learning by Toddlers.'' \emph{Cognition}. Elsevier.

\end{document}


